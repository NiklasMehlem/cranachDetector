%%%%%%%%%%%%%%%%%%%%%%%%%%%%%%%%%%%%%%%%%%%%%%%%%%%%%%%%%%%%%%%%%%%%%%%%%%%%%%%
% Globale Definitionen
%%%%%%%%%%%%%%%%%%%%%%%%%%%%%%%%%%%%%%%%%%%%%%%%%%%%%%%%%%%%%%%%%%%%%%%%%%%%%%%

%=== pdf Metadaten ============================================================
\hypersetup{
	pdfauthor={Vorname Nachname},
	pdftitle={Vorlage für wissenschaftliche Abschlussarbeiten an der TH Köln},
	pdfsubject={Abschlussarbeit},
	pdfkeywords={
		LaTeX,
		Abschlussarbeit,
		Vorlage,
	},
	bookmarksnumbered=true,
	pdfstartview=FitH,
	hidelinks,
}

%=== Vorwort vor Literatur ====================================================
\defbibnote{mynote}{%
%	Wie in \cref{sec:bib-content} erläutert, werden im Literaturverzeichnis 
%	ausschließlich die Quellen angegeben, auf die im Rahmen einer Arbeit 
%	tatsächlich verwiesen wird. Bitte prüfen Sie also das Literaturverzeichnis 
%	Ihrer Arbeit immer dahingehend, ob alle zitierten Quellen~--~und nur
%	diese~--~erfasst wurden. Dies trifft auf das nun folgende Verzeichnis
%	\emph{nicht} zu; die 
%	meisten der hier aufgeführten Quellen werden in dieser Vorlage nicht 
%	zitiert. Es handelt sich lediglich um ein Beispiel für ein 
%	Literaturverzeichnis mit Literaturempfehlungen zum wissenschaftlichen 
%	Schreiben.
}

%\bibliographystyle{apa}  % Oder z. B. alpha, ieee, apa, etc.
%\bibliography{references}  % Ohne .bib-Endung!
\addbibresource{references.bib}


%=== Kopf-/Fusszeile definieren ===============================================
\clearpairofpagestyles
\ohead[]{\headmark}
\ofoot[\pagemark]{\pagemark}

%=== Farben definieren ========================================================
\definecolor{THRed}{RGB}{207,24,32}
\definecolor{THOrange}{RGB}{236,101,37}
\definecolor{THPurple}{RGB}{175,54,140}

\definecolor{PetroffBlue}{RGB}{87,144,252}
\definecolor{PetroffOrange}{RGB}{248,156,32}
\definecolor{PetroffRed}{RGB}{228,37,54}

%=== Einstellungen für cref ===================================================
\newcommand{\crefpairconjunction}{ und~}
\newcommand{\crefrangeconjunction}{ bis~}
\crefname{figure}{Abbildung}{Abbildungen}

%=== Einstellungen für plots ==================================================
\pgfplotsset{
	compat=newest,
	/pgf/number format/.cd,
	dec sep={\text{,}},
	1000 sep={\,},
}

%=== Einstellungen für listings ===============================================
\lstdefinestyle{myLaTeX}{
	basicstyle=\footnotesize\ttfamily,
	language=TeX,
	keywordstyle=\color{blue},
	frame=single,
	backgroundcolor=\color{gray!10},
	tabsize=2,
	morekeywords={
		lstdefinestyle,
		footnotesize,
		ttfamily,
		color,
	},
}

\lstdefinestyle{myBasic}{
	basicstyle=\footnotesize\ttfamily,
	frame=single,
	escapechar={|_},
	backgroundcolor=\color{white},
	keywordstyle=\color{black},
}

\lstset{style=myLaTeX}

%\lstset{language=Python, basicstyle=\small, frame=single, numbers=left, xleftmargin=2em, framexleftmargin=1.5em}
%\renewcommand{\lstlistingname}{Algorithmus}


%=== paar convenience Sachen definieren =======================================
\DeclareMathOperator{\sgn}{sgn}
\newcommand{\vecW}{\ensuremath{\mathbf{w}}}%
\newcommand{\ci}{\ensuremath{\mathrm{i}}}%
%
\newcommand{\tb}{\textbackslash}
%
\newcommand{\comm}[1]{\enquote{\texttt{\tb #1}}}
%
\newcommand{\param}[1]{%
	$\langle$\textrm{\textit{#1}}$\rangle$%
}

\newcommand{\monofett}[1]{\textbf{\texttt{\seqsplit{#1}}}}

%=== Arial als Hauptschriftart ================================================
%\setsansfont{Arial}
%\renewcommand{\familydefault}{\sfdefault}
