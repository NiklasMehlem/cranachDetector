\chapter{Fazit}
%
Das Ziel der Arbeit, Gesichtserkennungsmodelle zu finden, welche für historische Gemälde geeignet sind, konnte größtenteils erfüllt werden. Das Modell RetinaFace alleine ist in der Lage, beinahe alle Gesichter auf getesteten Einzel-, Zwei- und Drei-Personen-Porträts zu finden. Zusammen mit \gls{mtcnn} und Dlib \gls{cnn} konnte ein Python-Modul entwickelt werden, welches in der Lage ist, alle Gesichter auf den getesteten Porträts zu markieren. Dieses bietet die Möglichkeit, Modelle je nach Bild an- oder abzuschalten sowie ihre \gls{confidence}-Grenzwerte anzupassen. Damit ist das Modul in der Lage, flexibel auf verschiedene Werke angewendet zu werden, um bestmögliche Ergebnisse zu erzielen. Des Weiteren konnten Funktionen implementiert werden, die prüfen, ob sich ein Overlay auf einem Gesicht befindet. Alternativ kann die ausgegebene Liste von Gesichtsbereichen verwendet werden, um eigene Funktionen zu implementieren. Sollen die Werte außerhalb von Python verwendet werden, kann die Liste zu einer JSON-Datei formatiert werden, um sie in anderen Anwendungen zu nutzen.

Als mögliche Weiterführung der Arbeit kann die GUI des Moduls überarbeitet werden, um die Nutzerfreundlichkeit zu optimieren. Ein mögliches Feature, das hinzugefügt werden könnte, wäre ein Zähler, welcher zeigt, wie viele Bereiche vom jeweiligen Modell markiert wurden. So kann schneller von Nutzer*innen erkannt werden, ob sich auf einem Bild False-Positives befinden, wenn die Zahl der Bereiche die tatsächliche Anzahl der Gesichter auf dem Bild übersteigt. Die dafür notwendigen Funktionen sind bereits implementiert, jedoch wurde aus Zeitmangel noch keine grafische Darstellung dafür umgesetzt. Auch könnte eine Funktion implementiert werden, welche die Liste der markierten Bereiche direkt zu einer JSON-Datei formatiert, um sie auch in anderen Programmen nutzen zu können.

Eine weitere mögliche Fortführung der Arbeit wäre ein intensiveres Testen mit Fokus auf Gruppenbilder. Die Tests an Gruppenbildern haben bereits gezeigt, dass es deutlich schwieriger ist, auf ihnen zuverlässig Gesichter zu finden. Aufgrund des beschränkten zeitlichen Rahmens wurden spätere Tests an Gruppenbildern verworfen, um den Fokus auf Porträts setzen zu können. Dies geschah mit der Überlegung, dass das Platzieren eines Overlays auf dem Gesicht eines Porträts die Ästhetik des Bildes stärker beeinflusst. Würde man die Forschung mit Gruppenbildern fortsetzen wollen, so müssten weitere Modelle, die nicht in dieser Versuchsreihe getestet wurden, herangezogen werden. Alternativ könnte in Erwägung gezogen werden, ein eigenes Modell für diesen Anwendungszweck zu trainieren oder zu entwickeln.