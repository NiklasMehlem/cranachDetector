\chapter{Fazit}
%
Das Ziel der Arbeit, Gesichtserkennungsmodelle zu finden, welche für historische Gemälde geigten sind konnte größtenteils erfüllt werden. Das Modell RetinaFace alleine ist in der Lage beinah alle Gesichter auf getesteten 1er, 2er und 3er Portraits zu finden. Zusammen mit \gls{mtcnn} und Dlib \gls{cnn} konnte ein Python-Modul entwickelt werden, welches in der Lage ist alle Gesichtern auf den getesteten Portraits zu finden. Mit der Möglichkeit Modelle je nach Bild an oder abzuschalten, sowie ihre confidence Grenzwerte anzupassen, ist das Modul in der Lage flexible auf verschiedene Werke angewendet zu werden um bestmögliche Ergebnisse zu Erzielen. Des weiteren konnten Funktionen implementiert werden die Nutzer*innen ermöglichen zu Prüfen ob sich ein Overlay auf einem Gesicht befindet. Mit der Möglichkeit eigene Funktionen zu implementieren in dem die ausgegebene Liste direkt in Python genutzt wird oder sie zum Beispiel zu einer JSON Datei formatiert um sie in anderen Anwendung zu nutzen.

Als mögliche Fortsetzungen zur Arbeit kann das GUI des Moduls überarbeitet um die Nutzerfreundlichkeit zu optimieren. Ein Mögliches Feature welches hinzugefügt werden könnte wäre ein Zähler, welcher Zeigt wie viele Bereiche von je welchem Modell markiert wurden. So kann auf die schneller von Nutzer*innen erkannt werden ob sich auf einem Bild False-Positives befinden, wenn die Zahl der Bereiche die der tatsächlichen Gesichter auf dem Bild übersteigt. Die dafür notwendigen Funktionen sind auch bereits implementiert, nur aufgrund von Zeitmangel keine grafische Darstellung für sie implementiert. Auch könnte eine Funktion implementiert werden, welche die Liste der markierten Bereiche direkt zu einer JSON Datei formatiert, um sie auch in anderen Programmen nutzen zu können.

Eine weitere mögliche Fortsetzung der Arbeit wäre ein intensiveres Testen mit Fokus auf Gruppenbildern. Gruppenbilder haben bereits in den Stichproben Tests gezeigt das es deutlich schwieriger ist auf ihnen zuverlässig Gesichter zu finden. Aufgrund des Mangels der Ressource Zeit wurden später Tests an Gruppenbildern verworfen um sich besser auf Portraits fokussieren zu können. Dies Geschah unter der Begründung das das Platzieren eines Overlays auf dem Gesicht eines Portraits die Ästhetik stärker negativ beeinflusst als auf einem Gruppenbild. Würde man die Forschung mit Gruppenbildern Fortsetzen wollen, so müssten vermutlich neue Modelle die nicht in dieser Versuchsreihe getestet wurden herangezogen werden, oder in Erwägung gezogen werden ein eigenes Modell für diesen Anwendungszweck zu trainieren.