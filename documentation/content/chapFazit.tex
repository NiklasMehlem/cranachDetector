\chapter{Fazit}
%
Das Ziel der Arbeit, Gesichtserkennungsmodelle zu finden, welche für historische Gemälde geeignet sind, konnte größtenteils erfüllt werden. Das Modell RetinaFace alleine ist in der Lage, beinahe alle Gesichter auf getesteten Einzel-, Zwei- und Drei-Personen-Porträts zu finden. Zusammen mit \gls{mtcnn} und Dlib \gls{cnn} konnte ein Python-Modul entwickelt werden, welches in der Lage ist, alle Gesichter auf den getesteten Porträts zu finden. Mit der Möglichkeit, Modelle je nach Bild an- oder abzuschalten sowie ihre \gls{confidence}-Grenzwerte anzupassen, ist das Modul in der Lage, flexibel auf verschiedene Werke angewendet zu werden, um bestmögliche Ergebnisse zu erzielen. Des Weiteren konnten Funktionen implementiert werden, die Nutzer*innen ermöglichen zu prüfen, ob sich ein Overlay auf einem Gesicht befindet. Mit der Möglichkeit, eigene Funktionen zu implementieren, indem die ausgegebene Liste direkt in Python genutzt wird oder sie zum Beispiel zu einer JSON-Datei formatiert wird, um sie in anderen Anwendungen zu nutzen.

Als mögliche Fortsetzung der Arbeit kann das GUI des Moduls überarbeitet werden, um die Nutzerfreundlichkeit zu optimieren. Ein mögliches Feature, das hinzugefügt werden könnte, wäre ein Zähler, welcher zeigt, wie viele Bereiche von welchem Modell markiert wurden. So kann schneller von Nutzer*innen erkannt werden, ob sich auf einem Bild False-Positives befinden, wenn die Zahl der Bereiche die tatsächliche Anzahl der Gesichter auf dem Bild übersteigt. Die dafür notwendigen Funktionen sind bereits implementiert, jedoch wurde aus Zeitmangel noch keine grafische Darstellung dafür umgesetzt. Auch könnte eine Funktion implementiert werden, welche die Liste der markierten Bereiche direkt zu einer JSON-Datei formatiert, um sie auch in anderen Programmen nutzen zu können.

Eine weitere mögliche Fortsetzung der Arbeit wäre ein intensiveres Testen mit Fokus auf Gruppenbilder. Gruppenbilder haben bereits in den Stichprobentests gezeigt, dass es deutlich schwieriger ist, auf ihnen zuverlässig Gesichter zu finden. Aufgrund des Mangels an Ressourcen (Zeit) wurden spätere Tests an Gruppenbildern verworfen, um sich besser auf Porträts fokussieren zu können. Dies geschah unter der Begründung, dass das Platzieren eines Overlays auf dem Gesicht eines Porträts die Ästhetik stärker negativ beeinflusst als auf einem Gruppenbild. Würde man die Forschung mit Gruppenbildern fortsetzen wollen, so müssten vermutlich neue Modelle, die nicht in dieser Versuchsreihe getestet wurden, herangezogen werden oder in Erwägung gezogen werden, ein eigenes Modell für diesen Anwendungszweck zu trainieren.