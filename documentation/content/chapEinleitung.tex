\chapter{Einleitung}
%
Gesichtserkennungsmodelle werden hauptsächlich mit Trainingsdaten bestehend aus Gesichtern von Realen Menschen entwickelt, und sind darauf ausgelegt Gesichter von Realen Menschen zu erkennen. Als für das Cranach Digital Archive neue Wasserzeichen entwickelt wurden, kam die Idee auf ob man diese Automatisiert auf den verschiedenen Bildern der Werke platzieren könnte, während Gesichter aus Ästhetischen gründen frei gelassen werden sollten. Hieraus entstand die Motivation dieser Arbeit. Das Testen und Vergleichen von bestehenden Gesichtserkennungsmodellen. Herauszufinden ob sich diese für den Einsatz auf historischen Gemälden eignen und falls ja, welche sich von ihnen am besten eignen. Hierfür wurden die Modelle von Haar-Cascade, Caffe, MediaPipe, Dlib HOG, Dlib Landmark, Dlib CNN, MTCNN, Yunet und RetinaFace verglichen. Aus ausgewählten geeigneten Modellen soll dann ein Python-Modul entwickelt werden, welches Bereiche von Gesichtern ausgibt. So soll die Möglichkeit gegeben werden zu überprüfen ob sich ein Overlay auf einem Gesicht befindet.

Im Recherche Kapitel wird kurz auf die getesteten Modellen eingegangen und ihre grobe Funktionsweise erläutert um sie ihre Ergebnisse in einen Kontext zu packen.
Als erster Schritt der Testreihe werden in den Stichprobentests die Verschiedenen Modelle auf ihre Eignung getestet. Es soll geschaut werden ob das getestete Modell überhaupt in der Lage ist mit historischen Gemälden zu arbeiten. Falls nicht werden sie bereits hier verworfen um in späteren Tests Zeit und Aufwand zu sparen.
Danach Folgen Tests mit Bildern verschiedener Auflösungen. So soll geschaut werden ob die Auflösung eines Bildes Auswirkungen auf die Ergebnisse eines Modells hat und welche dies sind. Auch soll so bestimmt werden welche Auflösung die Bilder der folgenden Tests haben sollen, um bestmögliche Ergebnisse zu ermöglichen.
Im vorletzten Test, sollen die confidence Grenzwerte der Modelle optimiert werden. So soll sichergestellt werden dass die bestmöglichen Ergebnisse der Modelle im finalen Test verglichen werden um eine möglichst sichere Entscheidung zutreffen welche Modelle für das implementierte Modul verwendet werden sollen.
Der Finale Test vergleicht die Flase-Negative, False-Positives und Anzahl an Gesichtern die nur vom jeweiligen Modell erkannt wurden. So soll festgestellt werden welche Modelle die sichersten sind, und ebenfalls welche diese ergänzen könnten um auf möglichste jeder Art von Bild alle Gesichter zu erkennen.