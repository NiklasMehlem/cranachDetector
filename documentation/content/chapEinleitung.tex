\chapter{Einleitung}
%
Gesichtserkennungsmodelle werden hauptsächlich mit Trainingsdaten bestehend aus Gesichtern realer Menschen entwickelt und sind darauf ausgelegt, Gesichter realer Personen zu erkennen. Als für das \gls{cda} neue Wasserzeichen entwickelt wurden, wurde die Idee entwickelt. Diese sollten dann automatisiert auf den verschiedenen Bildern der Werke zu platzieren werden, wobei Gesichter aus ästhetischen Gründen ausgespart werden sollten. Ziel dieser Arbeit ist das Testen und Vergleichen bestehender Gesichtserkennungsmodelle, um herauszufinden, ob sich diese für den Einsatz auf historischen Gemälden eignen – und falls ja, welche sich dafür am besten eignen. Hierfür wurden die Modelle \textit{Haar-Cascade}, \textit{Caffe}, \textit{MediaPipe}, \textit{Dlib HOG}, \textit{Dlib Landmark}, \textit{Dlib CNN}, \textit{MTCNN}, \textit{Yunet} und \textit{RetinaFace} verglichen. Aus den am besten geeigneten Modellen soll anschließend ein Python-Modul entwickelt werden, das die Bereiche der Gesichtern eines Bildes ausgibt. Mithilfe dieser Daten soll die Möglichkeit gegeben werden zu überprüfen, ob sich ein Overlay potenziell auf einem Gesicht befindet und so einen Automatisierungsprozess zu ermöglichen.

Im Recherche-Kapitel wird kurz auf die getesteten Modelle eingegangen und ihre grobe Funktionsweise erläutert, um ihre Ergebnisse in einen Kontext einordnen zu können.
Als erster Schritt der Testreihe werden in Stichprobentests die verschiedenen Modelle auf ihre grundsätzliche Eignung geprüft. Es soll festgestellt werden, ob ein getestetes Modell überhaupt in der Lage ist, mit historischen Gemälden zu arbeiten. Modelle, die diese Voraussetzung nicht erfüllen, werden bereits an dieser Stelle ausgeschlossen, um in späteren Tests Zeit und Aufwand zu sparen.
Im Anschluss daran werden Tests mit Bildern unterschiedlicher Auflösungen durchgeführt. Ziel ist es, zu untersuchen, ob die Bildauflösung einen Einfluss auf die Ergebnisse der Modelle hat und welche Auswirkungen dabei zu beobachten sind. Zudem soll auf dieser Basis festgelegt werden, mit welcher Auflösung die Bilder in den folgenden Tests verwendet werden sollen, um möglichst gute Ergebnisse zu erzielen.
Darauf folgend werden die Confidence-Grenzwerte der Modelle im nächsten Test optimiert. Dies soll sicherstellen, dass in der finalen Auswertung die jeweils besten Ergebnisse der Modelle miteinander verglichen werden können, um eine fundierte Entscheidung darüber zu treffen, welche Modelle für das zu entwickelnde Python-Modul verwendet werden sollen.
Der finale Test vergleicht die False-Negatives, False-Positives sowie die Anzahl der Gesichter, die jeweils nur von einem bestimmten Modell erkannt wurden. Ziel ist es, festzustellen, welche Modelle die zuverlässigsten Ergebnisse liefern und welche sich potenziell ergänzen lassen, um in möglichst allen Bildtypen eine vollständige Gesichtserkennung zu ermöglichen.
