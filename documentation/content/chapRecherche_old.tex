\chapter{Recherche}

Nicht alle Gesichtserkennungsmodelle sind in der Lage überhaupt zuverlässig Gesichtern in alten Gemälden zu erkennen. Bevor also zu viel Zeitaufwand entsteht alle möglichen Modelle intensiv zu testen und mit einander zu vergleichen werden ihnen Stichprobenartig jeweils 4 Bilder vorgelegt um einzuschätzen ob diese sich für weitere Test eignen.

ERKLÄRE WIE DIE STICHPROBEN TESTS FUNKTIONIEREN

\section{Haar-Cascade}
Haar-Cascade, genauer \monofett{haarcascade\_frontalface\_default.xml}, eignet sich in keiner Weise. Es scheitert sowohl darin zuverlässig das Gesicht in einem Portrait zu erkennen, als auch darin mehrere Gesichter in einem Gruppen Bild zu erkennen. Zudem kommt es oft zu false-positives. Teilweise werden dadurch mehr false-positives als tatsächliche Gesichter markiert. Haar Cascade hat nicht direkt einen 'confidence' Wert den man bearbeiten kann aber ein äquivalent, jedoch ist selbst mit verschiedenen Werten Haar Cascade nicht in der Lage zuverlässige Ergebnisse zuliefern und scheidet damit bereits hier für weitere Versuche aus.
Anzumerken ist ebenfalls das Haar-Cascade bei gleichen Motiv mit höherer Auflösung noch schlechter abschnitt, in dem es zu mehr false-positives kam.

\section{Caffe}
Caffe besteht den Stichproben Test. Verwendet wird in dieser Arbeit \monofett{res10\_300x300\_ssd\_iter\_140000.caffemodel} zusammen mit \monofett{deploy.prototxt} zur Definierung der Parameter der Netzwerkarchitektur. In 1er, 2er und 3er Portraits schafft es jeweils alle 'Hauptgesichter' zu erkennen. Zusätzlich ist es dabei auch sehr sicher ohne ein einziges false-positiv bei allen Stichproben Tests. Allerdings scheint es dafür mit kleineren Gesichtern, oder Gesichtern die nicht Hauptfokus des Portraits sind Probleme zu haben. So wurde im Gruppen Bild nicht ein Gesicht maskiert, auch kein false-positiv. Reduziert man den confidence Wert auf 0.2 so ist Caffe auch in der Lage alle Gesichter im 3er Portrait zu erkennen, weiter ohne false-positives. Das Gruppenbild hingegen bleibt der Schwachpunkt des Modells. Um Gesichter im Gruppenbild zu erkennen muss der confidence Wert so weit reduziert werden, dass es zu so vielen false-positives kommt das die Ergebnisse unbrauchbar werden. %Caffe hat keine veränderungen bei Höherauflösenden Bildern, nur das Gruppen Bild hat mit gleichem Confidence Wert nun mehr false positives, kann aber ignoriert werden da es schon zuvor schwierigkeiten damit hatte
\section{MediaPipe}
MediaPipe schafft es beim 1er Portrait das Gesicht sicher zu erkennen ohne false-positives. Dies ist aber auch der einzige Stichproben Test den MediaPipe besteht. Bei allen anderen Tests wird von MediaPipe nichts markiert, weder Gesichter noch false-positives. Da dieses Modell sehr ineffektiv für den gewünschten Kontext scheint wird es hier aussortiert und wird nicht weiter getestet.

\section{Dlib HOG}
Das \textbf{Histogram of Oriented Gradients} (HOG) basierte Gesichtsdetektionsmodell von dlib zeigt in den Stichproben Tests, dass es gut darin ist kleine Gesichter zu erkennen aber Probleme mit größeren hat. So hat es sehr große Probleme das Gesicht beim 1er Portrait zu erkennen, ohne dass man den confidence Wert ins unbrauchbare reduziert. Je nach confidence Wert kommt es auch noch zu kleinen Mengen (ca. 0 bis 3) von false-positives. Doch wurden die Gesichter im 2er, 3er und auch größten Teils im Gruppenbild gut erkannt. Das Modell weißt zwar offensichtliche Schwäche auf, allerdings sind die Ergebnisse gut genug das es weiter getestet wird. Es könnte sich als nützliche alternative für Bilder herausstellen bei denen andere Modelle Probleme haben die Gesichter zu erkennen.

\section{Dlib HOG Landmark}
Ähnlich wie das Dlib HOG Modell zuvor hat Dlib HOG Landmark, in dieser Arbeit \monofett{shape\_predictor\_68\_face\_landmarks.dat}, Probleme damit große Gesichter zu erkennen. Es schneidet gleich schlecht ab wie Dlib HOG beim 1er Portrait. Es ist bei den anderen Tests allerdings genauer gewesen und hatte weniger false-positives. Somit wird Dlib HOG Landmark ebenfalls weiter getestet.

\section{Dlib CNN}
Dlib CNN basierend auf der \textbf{Max-Margin Object Detection} (MMOD), genauer \monofett{mmod\_human\_face\_detector.dat}, ist das spürbar langsamste Modell von allen. Bereits bei relativ kleinen Bildern (998 x 1314) muss für mehrere Sekunden auf ein Ergebnis gewartet werden, während die anderen Modelle unter einer Sekunde bleiben. Das Ergebnis der Stichproben Tests ist dafür Fehlerlos. Bei 1er, 2er und 3er Portraits wurden alle Gesichter erkannt ohne false-positives. Nur beim Gruppenbild wurde kein Gesicht oder false-positiv erkannt. Mit eine so zuverlässigen Ergebnis wird Dlib CNN weiter getestet und verglichen. Am ende muss sich zeigen, dass die deutlich längere Bearbeitungszeit sich auch in deutlich besserer Zuverlässigkeit widerspiegelt um sie zu rechtfertigen.

\section{MTCNN}
\textbf{Multi-task Cascaded Convolutional Networks} (MTCNN) hat den Stichproben Test für 1er, 2er und 3er Portraits fehlerlos bestanden. Beim Gruppen Bild wurden bloß ein paar Gesichter erkannt, dabei sollten nur Ergebnisse mit > 0.5 confidence makiert werden. Es gab des weiteren nicht einen false-positiv. Somit wird MTCNN sicher weiter getestet.

\section{Yunet}
Das Yunet Modell, in diesem fall \monofett{face\_detection\_yunet\_2023mar.onnx}, ist ohne weitere Einstellungen sehr streng. Bei den Stichproben Test kam es zu keinen false-positives, allerdings wurde in manchen Bildern nicht alle Gesichter markiert. Nachdem der confidence Wert auf 0.8 eingestellt wurde, wurde Problemlos alle Gesichter des 1er, 2er und 3er Portraits erkannt. Aus dem Gruppen Bild wurden nur wenige Gesichter markiert. Ob 0.8 der optimale Wert ist wird sich zeigen wenn Yunet intensiver getestet wird. Aufgrund seiner Genauigkeit wird Yunet weiter getestet.

\section{RetinaFace (CPU)}
Verwendet wird in dieser Arbeit RetinaFace mit CPU Einsatz. Grund dafür ist, dass RetinaFace mit GPU Einsatz lediglich schneller und nicht genauer sein soll. So bleiben die Werte von RetinaFace später vergleichbarer. Die Option das mit einer geeigneten GPU schnellere Ergebnisse erzielt werden können, kann später bei der Auswertung noch berücksichtigt werden. Was die Ergebnisse des Stichproben Tests angeht hat RetinaFace das beste Ergebnisse aller Modelle bisher. Alle Gesichter ohne false-positives im 1er, 2er und 3er Portrait erkannt. Noch dazu wurden 10 der 13 Gesichter im Gruppenbild erkannt, wo die besten Ergebnisse bisher ~5 Gesichter waren. Jedoch sei anzumerken das es 2 false-positives gab bei denen die Gesichter von Pferden markiert wurden. Ob diese RetinaFace diese Quote beibehalten kann wird sich in den folgenden Tests ergeben.