\chapter{Implementierung}

Ursprünglich war die Idee ein Eigenständiges Programm zu entwickeln, welches Selbstständig Wasserzeichen auf Bildern platziert. Dabei sollten die Gesichter freigelassen werden um die Ästhetik des Bildes nicht zu stören. Aufgrund der Limitierten Zeit wird dieser Ansatz reduziert, was ebenfalls ermöglicht ihn flexibler zu gestalten. Indem statt eines eigenständigen Programms ein Package erstellt wird, ermöglicht man noch mehr Anwendungszwecke über das einfache Platzieren von Wasserzeichen hinaus.

Um möglichst viele Verschiedene Anwendungszwecke abzudecken akzeptiert das Paket verschiedene Arten von eingaben:
\begin{itemize}
  \item None: Lässt den Nutzer einen Ordner in einem GUI Manuell wählen durch den Iteriert werden soll.
  \item File: Es kann Pfad zu einer einzelnen Bild Datei hinterlegt werden welche bearbeitet werden soll.
  \item Path: Wird stadessen ein Pfad gegeben zu einem Ordner, so werden alle Bilder dieses Ordners zu einer Liste verarbeitet durch die dann iteriert wird.
  \item List: Wenn die interne Funktion unzureichend ist für den Anwendungskontext oder man einfach bereits eine List mit Bilderpfaden hat so wird diese direkt verwendet.
\end{itemize}
